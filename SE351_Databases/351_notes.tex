\documentclass{article} 
\newcommand\tab[1][0.5cm]{\hspace*{#1}}

\title{SOFTENG351 Notes 2017} 
\author{Theodore Oswandi} 

\usepackage[
	lmargin=2.5cm,
	rmargin=7cm,
	tmargin=1cm,
	bmargin=3cm,
	]{geometry}
\usepackage{enumitem}
\setlist{noitemsep}
\usepackage[none]{hyphenat}

\begin{document} \maketitle{} 

\section{Fundamentals of Database Systems}
	\subsection{General Information}
		\paragraph{Database}
		large integrated collection of data. 
		\\ Contains [Entities, Relationships] 

		\paragraph{DBMS}(Database Management System): 
		\\ software package to store and manage databases

		\paragraph{Database System}: DBMS with database

		\paragraph{DBMS and uses}
		\begin{itemize}
			\item store large amounts of information
			\item code for queries
			\item protect from inconsistencies and crashes
			\item security
			\item concurrent access
		\end{itemize}

	\subsection{Why Databases}
		\tab Need to shift from computation to storage of large amounts of information
		\\ \\ \tab Accomodate for changes in:
		\\ \tab \textbf{Variety:} types of data
		\\ \tab \textbf{Velocity:} movement of data
		\\ \tab \textbf{Veracity:} uncertainty of data
		\\ \tab \textbf{Volume:} amount of data

		\paragraph{Structures/Models}
		Need to have a model to describe data, and a schema used to give an abstract description of the data model

	\subsection{Levels of Abstraction}
		\tab \textbf{Views:} describe how data seen
		\\ \tab \textbf{Logical Schema:} how data structures organised (variable types)
		\\ \tab \textbf{Physical Schema:} how files structured
		\\ \tab \textbf{Data Definition Language:} How to define database schema
		\\ \tab \textbf{Data Manipulation:} how to update values in database 
		\\ \tab \textbf{Query Language:} used to access data 

	\subsection{Data Independence}
		\tab \textbf{Logical Data Independence}
		\begin{itemize}
			\item external handling separate from logical organisation
			\item mappings change, not external schema
			\item applications only see external schema
		\end{itemize}
		\tab \textbf{Physical Data Independence}
		\begin{itemize}
			\item changes to physical schema doesn't affect logical layer
			\item abstract from DBMS storage organisation
			\item can perform optimisation/tuning
		\end{itemize}

	\subsection{Concurrency Control}
		\begin{itemize}
			\item many users have to be able to access information at the smae time and make updates without negatively affecting database
			\item don't want to access disk lots. It is slow and inefficient
			\item let multiple users access and keep data consistent
			\item let users feel like they're the only ones using system
		\end{itemize}

\section{Relational Model of Data}
	\subsection{General Information}
		\begin{itemize}
			\item is logical model of data
			\item distinguish between data syntax and semantics
			\item simple and powerful
			\item sql based off this
		\end{itemize}

	\subsection{Simple approach}
		\begin{itemize}
			\item use tuples to store data
			\item relations are sets of these tuples
			\item tables to represent sets of data
			\item properties (columns) are called attributes
			\item attributes associated with domains (variable types)
		\end{itemize}

	\subsection{Relational Schemata}

		Use of attributes creates relation schema such as:
		\\ MOVIE(title: $string$, production\_year: $number$)
		\\ \\
		\textbf{Relation Schema} provide abstract description of tuples in relation
		\\ \\
		\textbf{Database Schema} is set $S$ of relational schemata. Is basically the set of all tables and their attributes

	\subsection{Keys}
		Are used to uniquely identify tuples over all data in a given table.
		\\ They are used to restrict number of database instances, to something more realistic and identify objects efficiently
		\\ \\ 
		\textbf{Superkey} over relation schema is a subset of attributes that satisfies this uniqueness property
		\\ \textbf{Key} is a minimal superkey, is key if no other superkeys exist for R
		\\ \textbf{Foreign Key}: is a key used to index values from other values. Used to make reference between relational schemata. 
		\\ \tab - ensures referential integrity
		\\ \tab - no need to copy info from other tables
		\\ \tab - need to ensure that [x,y] $\subseteq$ [x,y] and not [y,x] (Order matters)
		\\ \\ \textbf{Example}
		\\ MOVIE(title: string, production\_year: number, director\_id: number) 
		\\ with key [title, production\_year]
		\\ \\ DIRECTOR(id: number, name: string)
		\\ with key [id]
		\\ \\ with foreign key: MOVIE[director\_id] $\subseteq$ DIRECTOR[id]

	\subsection{Integrity Constraints}
		\begin{itemize}
			\item Db schema should be meaningful, and satisfy all constraints
			\item should stay true to keys, and foreign keys
			\item constraints should interact with each other correctly
			\item should process queries and update efficiently
			\item should do this and make as few comprimises as possible
		\end{itemize}

\section{SQL as Data Definition and Manipulation Language}
	\subsection{General Information}
		\begin{itemize}
			\item \textbf{S}tructured \textbf{E}nglish \textbf{QUE}ry \textbf{L}anguage
			\item used as a standardised language in relational Db systems
			\item is query, data definition, and data management language
			\item names not case sensitive
		\end{itemize}

	\subsection{Keywords}
		\paragraph{CREATE} used to create tables. [CREATE TABLE $<tablename> <attribute specs>$]
		\paragraph{Attributes}: defined as [$<attribute\_name> <domain>$]
		\paragraph{NOT NULL}: used to ensure that attribute is not null.
		\paragraph{DROP TABLE}: removes relation schemata from Db system
		\paragraph{ALTER TABLE}: change existing relation schemata
		\\ \\ \textbf{CHARACTER} and \textbf{VARCHAR} are fixed or variable length strings. Variable length string are $VARCHAR(32)$
		\paragraph{NATIONAL CHARACTER}: string characters from other languages
		\\ \\ \textbf{INTEGER} and \textbf{SMALLINT} used for integers
		\\ \\ \textbf{NUMERIC} and \textbf{DECIMAL} used for fixed point numbers
		\\ \\ \textbf{FLOAT, REAL} and \textbf{DOUBLE PRECISION} used for floating point numbers
		\\ \\ \textbf{DATE} used for dates (year, month, day)
		\\ \\ \textbf{TIME} used for times (hour, minute, second)

	\subsection{Null and Duplicate Tuples}
		\paragraph{NULL} used to insert tuple where some of the information is not known. This is not good as it can create inconsistencies in information. SQL does this through use of a null marker

		\paragraph{Duplicate Tuples} are when another tuple exist with same values. NULL can make this confusing.
		\\ \tab - these duplicated really expensive to remove
		\\ \tab - relations do not contain these tuples

	\subsection{SQL Semantics}
		\paragraph{Table Schema} set of attributes
		\paragraph{Table over R} set of partial tuples
		\\ \\ These relations are idealised with no duplicates

	\subsection{Constraints}
		\begin{itemize}
			\item can add uniqueness parameters like primary keys on table
			\item NOT NULL makes sure that partial tuples don't exist
			\item \textbf{UNIQUE} $<attribute\_list>$ ensures that values for attributes exist only once
			\item \textbf{PRIMARY KEY} used on tables that are also UNIQUE, and ensures that NOT NULL applied to it too
			\item \mbox{$FOREIGN KEY <attributelist> REFERENCES <tablename> <attributelist>$}  used to create foreign key dependencies
			\item $CONSTRAINT <name> <constraint>$ \\used to give names to constraints
		\end{itemize}

	\subsection{Referential Actions}
		Are used to specify what happens if tuples updated or deleted
		\\ \\ SQL provies:
		\begin{itemize}
			\item \textbf{CASCADE} forces refercing tuples to be deleted
			\item \textbf{SET NULL} sets referencing tuples to NULL
			\item \textbf{SET DEFAULT} sets the tuple values to specified default
			\item \textbf{NO ACTION} does nothing
		\end{itemize}

		\paragraph{Example usage:}
		\mbox{FOREIGN KEY(title, year) REFERENCES MOVIE(title, year) ON DELETE \textbf{CASCADE}} \\ will delete director tuple if movie deleted
		\\\\ \mbox{FOREIGN KEY(title, year) REFERENCES MOVIE(title, year) ON DELETE \textbf{SET NULL}} \\ will keep tuple in director but won't know what he directed if movie deleted

	\subsection{CREATE statement}
		\begin{itemize}
			\item \textbf{CREATE ASSERTION} $<name>$ CHECK defines integrity constraints
			\item \textbf{CREATE DOMAIN} define domains and check clause
			\item \textbf{CREATE VIEW $<name> \textbf{AS} <query>$} define views
			\item \textbf{CREATE GLOBAL TEMPORARY TABLE} makes a table for current SQL session
			\item \textbf{CREATE LOCAL TEMPORARY TABLE} makes table for current module or session
		\end{itemize}

	\subsection{Use as Data Manipulation Language}
		Allow insert, delete or update

		\begin{itemize}
			\item INSERT INTO $<tablename>$ VALUES $<tuple>$
			\item INSERT INTO $<tablename> <attributelist>$ VALUES $<tuples>$
			\item INSERT INTO $<tablename> <query>$
			\item DELETE FROM $<tablename>$ WHERE $<condition>$
			\item \mbox{UPDATE $<tablename>$ SET $<value-assignment-list>$ WHERE $<conditions>$}
		\end{itemize}

\section{Relational Query Languages: Algebra}
	\subsection{Query Languages}
		\begin{itemize}
			\item allow access and retrieval of data
			\item foundation based on logic, and allows for optimisation
			\item are not programming languages
			\item not intended for complex claculations
			\item easy efficient access to lots of data
		\end{itemize}

		\textbf{Query Language} must have
		\begin{itemize}
			\item formal language with
			\item input schema
			\item output schema
			\item query mapping
		\end{itemize}

		\textbf{Equivalent Queries}
		\begin{itemize}
			\item same input schema
			\item same output schema
			\item same query mapping
			\item if L $\sqsubseteq$ L' and L' $\sqsubseteq$ L
		\end{itemize}

		\textbf{Dominant}, if Q1 dominates Q2
		\begin{itemize}
			\item if L $\sqsubseteq$ L'
		\end{itemize}

	\subsection{Relational Algebra}
		\begin{itemize}
			\item useful for representing execution plan
			\item easily translated to SQL
			\item A is set of possible relations
			\item Set of Operations:
				\begin{itemize}
					\item Attribute Selection $\sigma$
					\\ \tab returns value where A in MOVIE is same as B in MOVIE 
					\\ \tab $[\sigma _{A=B}(MOVIE)]$

					\item Constant Selection $\sigma$
					\\ \tab returns vlaue where A is 3 in MOVIE 
					\\ \tab $[\sigma _{A=3}(MOVIE)]$

					\item Projection $\pi$
					\\ \tab returns title and date of MOVIE 
					\\ \tab $[\pi _{title, date}(MOVIE)]$

					\item Renaming $\delta \mapsto$
					\\ \tab renames attribute in MOVIE 
					\\ \tab $[\delta _{title \mapsto title'}(MOVIE)]$

					\item Union $\bigcup$
					\\ \tab takes 2 arguments and returns set of tuples contained in either result, removing duplicates
					\\ \tab $[\sigma _{A=2}(MOVIE) \bigcup \sigma_{A=3}(MOVIE)]$

					\item Difference $-$
					\\ \tab returns negation
					\\ \tab $[-\sigma _{A=3}(MOVIE)]$

					\item Natural Join $\bowtie$
					\\ \tab Joins two tables and return tuples that match in both tables
					\\ \tab $[ MOVIE \bowtie DIRECTOR ]$
				\end{itemize}
		\end{itemize}
		Redundant Operations
		\begin{itemize}
			\item Intersection $\bigcap$
			\\ \tab because can use DeMorgans to produce with $- and \bigcup$

			\item Cross Product $\times$
			\\ \tab because creates $N\times M$ tuples which is not space efficient

			\item Division $\div$
			\\ \tab because it uses cross product to generate all those that match
		\end{itemize}

		$\pi \sigma \bowtie \delta \bigcap \bigcup \bigvee \bigwedge \div \times - \in \mapsto$

	\subsection{Query Language $\mathcal{L} _{ALG}$ of Relational Algebra}
		wtf is this

	\subsection{Incremental Query Formulation}
		Can use multiple Queries to break up calculation to make it more readable. Ths will also reduce the amount of copying needed as you can use previous results in subsequent calculations.

\section{Relational Query Language: Calculus}
	\subsection{General}
		\begin{itemize}
			\item let users describe what they want
			\item simple to write query and translate to SQL
			\item sometimes relational algebra can get convulated
			\item based on first order logic
			\item if safe then can automatically be transformed to SQL style
			\item \textbf{Tuple Relational Calculus TRC} variables represent tuples
			\item \textbf{Domain Relational Calculus DRC} variables represent values in domain
		\end{itemize}

		\textbf{Query Structure \& Example}
		\\ \tab $\{ \tab[0.2cm] to\_return \tab[0.2cm]|\tab[0.2cm] \exists x,y,z(sometable(x,y,z,n)\}$
		\\ \tab $\{x_{name} | \exists x_{age} (PERSON(x_{name}, x_{age}, 'Male'))\}$

	\subsection{Object Properties}
		have variables or may set its value in query
		variable values must be in its domain
		can make complex properties by using negation, conjunction and exisential quantification
		remove brackets if doesn't increase ambiguity

	\subsection{Shortcuts}
		\begin{itemize}
			\item \textbf{Inequation} $A \neq B$ equivalent to $\neg A = B$
			\item \textbf{Disjunction} DeMorgan's: $A \vee B$ shortcuts $\neg(\neg A \wedge B)$
			\item \textbf{Universal Quantification} DeMorgan's: $\forall x (A)$ shortcuts $\neg \exists x (\neg A)$
			\item \textbf{Implication} $A \Rightarrow B $ shortcuts $ \neg A \vee B$
			\item \textbf{Equivalence} $A \Leftrightarrow B$ shortcuts $(A\Rightarrow B) \wedge (B \Rightarrow A)$ 
			\item \textbf{Successive Exestential Quantification} $\exists x_1 ( \exists x_2 (A))$ same as $\exists x_1,x_2(A)$
			\item \textbf{Successive Univeral Quantification} $\forall x_1 ( \forall x_2 (A))$ same as $\forall x_1,x_2(A)$
		\end{itemize}

	\subsection{Free variables}
		\begin{itemize}
			\item placeholders used to describe structure
			\item basically doesn't have to hold a specific set of values
			\item not bound to any quantifier
			\item negation does nothing to free variables
			\item obeys conjunction
		\end{itemize}

	\subsection{Formulae interpretation}
		\tab Whole thing either evaluates to T or F

	\subsection{Domain Relational Calculus}
		\begin{itemize}
			\item Truth value of formula based on its values on free variables
			\item \textbf{Example:} $\{ (m) | \exists d, t(SCREENING ('Event', m, d, t)) \}$
			\item if looking for empty set, then just returns if value exists for that or not
		\end{itemize}

	\subsection{Tuple Relational Calculus}
		\begin{itemize}
			\item named on global perspective
			\item sort logic based on domain and relation schemata
			\item formula built fromatms on DRC. Free variables defined analogously
		\end{itemize}

	\subsection{Safe Range Normal Form}
		\begin{itemize}
			\item check if query is safe or not
			\item basically don't want potentially infinite number of values

			\item Must contain these properties:
				\begin{enumerate}
					\item \textbf{eliminate shortcuts} (Universal Quantification $\forall$, Implication $\Rightarrow$ and Equivalence $\Leftrightarrow$)
					\item \textbf{bounded renaming}: ensure no free and bound variables.\\ \tab Not allowed: $\neg (A \vee B)$
					\item remove any double negations $\neg \neg A$ to $A$
					\item only put negation in front of atom or quantifier, and not over bracketed area
					\item omit unecessary brackets
					\item ensure there is no disjunction in final formula
				\end{enumerate}

			\item if all free variables range restricted then formula safe
			\item if in SRNF then everything should either be an atom or quantified formula
		\end{itemize}

		% $\neq \Leftrightarrow \Leftarrow \Rightarrow \forall \exists \neg x \vee \wedge$


\section{SQL: As a Query Language}
	\textbf{Basic structure:}
		\\\tab$SELECT <attribute list> FROM <table list> WHERE <condition>$
	\\\\\textbf{Renaming} (tables and attributes):
		\\\tab$<name> AS <new name>$
	\\\\\textbf{Select and remove duplicates}
		\\\tab$SELECT DISTINCT \dots$
	\\\\\textbf{Sort results} (can sort by multiple in heirarchy):
		\\\tab $ORDER BY <attribute list>$
		\\\tab add $ASC$ or $DESC$ to determine if want increasing/decreasing

	\subsection{Operators:}
		\tab$=, <, >, \leq, \geq, <>$(not equal to)

	\subsection{Where Structure}
		\tab$[<expression> <operator> <expression>]$

	\subsection{Values in a range}
		\tab$[<attribute> BETWEEN <value> AND <value>]$
		\\\tab$[<attribute> IN <value list>]$

	\subsection{Aggregate functions}
		\begin{itemize}
			\item \textbf{COUNT}: returns number of values
			\item \textbf{AVG}: return average argument values. Only work if domain some kind of number
			\item \textbf{MIN}: returns minimum argument value
			\item \textbf{MAX}: returns maximum argument value
			\item \textbf{SUM}: returns sum of argumet values. Only works if domain is number.
		\end{itemize}

		Results of above functions different if distinct used

	\subsection{Grouping}
		\begin{itemize}
			\item used when only want to apply aggregate on group of tuples
			\item use $GROUP BY <attribute list>$
			\item if want to only do some groups then add  $HAVING <condition>$
			\item the HAVING condition must apply to group and not tuples in groups
		\end{itemize}

	\subsection{Subqueries}
		\begin{itemize}
			\item can utilise further queries in the WHERE conditions
			\item $\textbf{IN} <subquery>$ checks if tuple exists in subquery
			\item $\textbf{EXISTS} <subquery>$ checks if subquery returns non empty relation
			\item $\textbf{UNIQUE} <subquery>$ checks if result contains no duplicates
			\item can \textbf{NOT} above queries if you want
		\end{itemize}

	\subsection{Reachability}
		\begin{itemize}
			\item basically seeing if given a graph if you can go from A to X in $n$ number of stops
			\item If n is known then can use iteration
			\item However if n is flexible then must use recursion somehow
		\end{itemize}

	\subsection{Recursion}
		wat.

	\subsection{Writing Difficult SQL Query}
		\begin{enumerate}
			\item Formalise query in safe calculus
			\item Transform query into SRNF
			\item Transform SRNF to SQL
		\end{enumerate}

	\subsection{Division}
		Basically if $R \div S$, then return all attributes in R that satisfy all values in S, except columns in S.
		utilises 2 nested WHERE NOT EXISTS

	\subsection{Outer Joins}
		\begin{itemize}
			\item used when $t_1 \in R_1$ don't match $t_2 \in R_2$. So can't just do $R_1 \bowtie R_2$
			\item \textbf{FULL OUTER JOIN}\\\tab
			if $r$ FULL OUTER JOIN $s$ then do join, keep all information from both tables and make unknown attributes NULL
			\item \textbf{LEFT OUTER JOIN}\\\tab
			if $r$ LEFT OUTER JOIN $s$, then do the join, and keep entirety of $r$, those values not in $s$ make resulting column for that tuple NULL
			\item \textbf{RIGHT OUTER JOIN}\\\tab
			if $r$ RIGHT OUTER JOIN $s$, then do join and keep entirety of $s$ but only match $r$ if exists. Unknown values become NULL
		\end{itemize}


	\textbf{Possibly make list of examples here}

\section{Entitiy Relationship Model}
	\subsection{Database Design}
		\subsubsection{General Information}
			\begin{itemize}
				\item Organisations now have access to lots of data
				\item Data is recorded in databases
					\begin{itemize}
						\item What to keep in Db?
						\item How to access it?
					\end{itemize}
				\item Database design aims to create and manage complex databases for huge information systems
				\item Broken down into four phases
					\begin{enumerate}
						\item Requirement Analysis
						\item Conceptual Design
						\item Logical Design
						\item Physical Design
					\end{enumerate}
			\end{itemize}

		\subsubsection{Conceptual Design}
			\paragraph{target of database} meet organisation that is going to use database needs
			\paragraph{conceptual design} provides abstract description of database in terms of high level concepts. Shows expected structure
			\paragraph{Inputs} are information requirements of users
			\paragraph{Output} is database schema, with consideration of user requirements but no layout, implementatation and phyiscal details yet
			\paragraph{Conceptual data model} to describe language

	\subsection{Entity Relationship}
		\subsubsection{Entities}
			\begin{itemize}
				\item basic objects
				\item described by attributes
				\item these attributes should allow differentiation between objects
				\item \textbf{key} created from these attributes to uniquely identify objects
			\end{itemize}

			Look
			\begin{itemize}
				\item Visualised by rectangles
				\item Attributes point to rectangle
				\item Attriutes that create key are underlined
			\end{itemize}

		\subsubsection{Relationships}
			\begin{itemize}
				\item create connection between various entities
				\item are also technicalle objects that connect other objects together
			\end{itemize}

			\tab \textbf{Format:} {COMPONENTS, ATTRIBUTES, KEYS}
			\\ \tab \tab \textbf{Examples:} $\{\{Student:PERSON\},\{Year, Course\}, \{Student, Year, Course\}\}$

			\textbf{Look}
			\begin{itemize}
				\item Visulaised by diamonds
				\item Contains attributes as well, but doesn't have to.
				\item Same as Entities, key attributes are underlined
				\item Edged linked to it are objects it associates with
				\item Components that are part of key have a dot in the line connecting it to relationship diamond
				\item key may span all attributes
			\end{itemize}

			\textbf{Roles}
			\begin{itemize}
				\item used to tie together relationship containing at least 2 of hte same object type. Both will have a line connecting diamond and rectangle, each named respectively.
				\item effectively make use of foreign keys
			\end{itemize}
			\tab \textbf{Example:} $\{\{Student:PERSON, Lecturer:PERSON\},\emptyset, \{Student, Lecturer\}\}$

		\paragraph{unique-key-value property} exists due to attributes in object uniquely identifying tuples. In a given set of tuples no duplicates in attribute subset that makes up key will exist.

	\subsection{Set vs Foreign Key Semantics}
		\textbf{Set semantics} uses entire attribute set\\
		\textbf{Foreign key semantics} uses only key attributes
		\\\\If using foreign key semantics then definitions of [Relationships, Relationship Sets, Database Instances] are slightly different. But also lets you use $E_{id}$ instead of having to use entire $E$ to reference things.
		\\\\Due to unique-key-value property, using either $E$ or $E_{id}$ is equivalent since can identify tuple by key attributes

	\subsection{Identifiers}
		For given Entity, can create new attribute that identifies rest of attributes for given tuple. If can ensure this column is unique then can use it as single attribute key. This kind of associates an attribute with an entity

		\paragraph{unique identifier property} bascally saying that identifiers that identify relationships must be unique

	\subsection{Specialisation, Generalisation, Clusters}
		\subsubsection{Specialisation}
			\begin{itemize}
				\item lets you define multiple types of an entity. Basically subtypes
				\item Such as Student, Lecturer, Tutor which are all people
				\item Generally adds various attributes that relate to specialised object type.
				\item Is treated as a relation
				\item Can have specialisations of specialisations (relationship of relationship)
			\end{itemize}

			Subtype U with supertype C:\tab\tab $U = (\{C\}, attr(U),\{C\}$

		\subsubsection{Generalisation}
			\begin{itemize}
				\item abstract concepts that model multiple things
				\item allows you to create a single relationship for connections
			\end{itemize}

		\subsubsection{Clusters}
			\begin{itemize}
				\item model disjoint union (or)
				\item ensure that things are mutually disjoint
				\item attach those cluster groups to a single point $\oplus$
				\item cluster still labelled as an object
			\end{itemize}

	\subsection{Transforming the ER diagram}
		\begin{itemize}
			\item these diagrams can be transformed to database schemata for things such as importing into SQl for example.
			\item Entities become tables
			\item Relationships become tables with foreign key connections to other tables
		\end{itemize}

	\subsection{Handling Clusters}
		\begin{itemize}
			\item clusters in conceptual design to model alternatives
			\item to handle them, must remove them from ER diagram to see individual interactions.
			\item cluster provide convenient way to model objects to Db
			\item if you avoid clusters then ER schemata becomes large
		\end{itemize}

\section{Database Design Quality: Relation Normalisation}
	\subsection{General}
		\begin{itemize}
			\item \textbf{Normalisation}'s goal is to efficiently process updates
				\begin{itemize}
					\item this aims to have no data redundancy
					\item redundancy means you have to do many updates
					\item also means you have to keep checking that data integrity exists and all redundant values updated as needed.
				\end{itemize}
			\item \textbf{Denormalisation}'s goal is to efficiently process queries
				\begin{itemize}
					\item because joins are expensive to do
				\end{itemize}
			\item cannot always do this as they trade off each other
			\item therefore try to design datbase well 
		\end{itemize}

	\subsection{Functional Dependencies}
		\begin{itemize}
			\item if $X \rightarrow Y$ then for a given value of X, the Y value will be same.
			\item therefore if $Student \rightarrow Teacher$ then that student will always have that teacher
			\item if $\emptyset \rightarrow X$ then X can only have one value
			\item attributes occuring in a key is a \textbf{prime attribute}

		\end{itemize}

	\subsection{Derivation Rules}
		General formula \tab $\frac{IF}{THEN}$
		\\\\Transitivity\tab$\frac{X \rightarrow Y, Y \rightarrow Z}{X \rightarrow Z}$
		\\\\Extension\tab$\frac{X \rightarrow Y}{X \rightarrow XY}$
		\\\\Reflexivity\tab$\frac{}{XY \rightarrow Y}$
		\\\\Also something about derivation trees but I don't understand

	\subsection{Soundness \& Completeness}
		\paragraph{Sound} R is sound if every derivable dependency is implied
		\paragraph{Complete} R is complete if every implied dependency is derivable

	\subsection{Canonical Cover}
		Get all functional dependencies and create a minimal list that still preserves its properties
		\\\\\tab Contains:
		\begin{enumerate}
			\item Get all Functional Dependencies
			\item \textbf{Decompose}: break $A \rightarrow BC$ to $A \rightarrow B,  \rightarrow C$
			\item \textbf{L-reduced cover} which is when you do $AB \rightarrow C$ to $A \rightarrow C$
			\item Then you are left with canonical cover when you delete duplicates and redundant dependencies.
		\end{enumerate}

	\subsection{Boyce-Codd Normal Form}
		\begin{itemize}
			\item can always get lossless BCNF decomposition
			\item guarantee no redundancy in terms of FD
			\item might exist some FD that can't be enfored
		\end{itemize}

	\subsection{Third Normal Form}
		\begin{itemize}
			\item Lossless 3NF can always be achieved
			\item guarantees that all FDs can be enforced on elements in 3NF decomposition
			\item ensures least level of data redundancy among all lossless, faithful decompositions
		\end{itemize}

\end{document}