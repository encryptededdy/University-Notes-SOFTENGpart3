\documentclass{article} 
\newcommand\tab[1][0.5cm]{\hspace*{#1}}

\title{SOFTENG351 Notes 2017} 
\author{Theodore Oswandi} 

\usepackage[
	lmargin=2.5cm,
	rmargin=7cm,
	tmargin=1cm,
	bmargin=3cm,
	]{geometry}
\usepackage{enumitem}
\setlist{noitemsep}
\usepackage[none]{hyphenat}

\begin{document} \maketitle{} 

\section{Fundamentals of Database Systems}
	\subsection{General Information}
		\paragraph{Database}
		large integrated collection of data. 
		\\ Contains [Entities, Relationships] 

		\paragraph{DBMS}(Database Management System): 
		\\ software package to store and manage databases

		\paragraph{Database System}: DBMS with database

		\paragraph{DBMS and uses}
		\begin{itemize}
			\item store large amounts of information
			\item code for queries
			\item protect from inconsistencies and crashes
			\item security
			\item concurrent access
		\end{itemize}

	\subsection{Why Databases}
		\tab Need to shift from computation to storage of large amounts of information
		\\ \\ \tab Accomodate for changes in:
		\\ \tab \textbf{Variety:} types of data
		\\ \tab \textbf{Velocity:} movement of data
		\\ \tab \textbf{Veracity:} uncertainty of data
		\\ \tab \textbf{Volume:} amount of data

		\paragraph{Structures/Models}
		Need to have a model to describe data, and a schema used to give an abstract description of the data model

	\subsection{Levels of Abstraction}
		\tab \textbf{Views:} describe how data seen
		\\ \tab \textbf{Logical Schema:} how data structures organised (variable types)
		\\ \tab \textbf{Physical Schema:} how files structured
		\\ \tab \textbf{Data Definition Language:} How to define database schema
		\\ \tab \textbf{Data Manipulation:} how to update values in database 
		\\ \tab \textbf{Query Language:} used to access data 

	\subsection{Data Independence}
		\tab \textbf{Logical Data Independence}
		\begin{itemize}
			\item external handling separate from logical organisation
			\item mappings change, not external schema
			\item applications only see external schema
		\end{itemize}
		\tab \textbf{Physical Data Independence}
		\begin{itemize}
			\item changes to physical schema doesn't affect logical layer
			\item abstract from DBMS storage organisation
			\item can perform optimisation/tuning
		\end{itemize}

	\subsection{Concurrency Control}
		\begin{itemize}
			\item many users have to be able to access information at the smae time and make updates without negatively affecting database
			\item don't want to access disk lots. It is slow and inefficient
			\item let multiple users access and keep data consistent
			\item let users feel like they're the only ones using system
		\end{itemize}

\section{Relational Model of Data}
	\subsection{General Information}
		\begin{itemize}
			\item is logical model of data
			\item distinguish between data syntax and semantics
			\item simple and powerful
			\item sql based off this
		\end{itemize}

	\subsection{Simple approach}
		\begin{itemize}
			\item use tuples to store data
			\item relations are sets of these tuples
			\item tables to represent sets of data
			\item properties (columns) are called attributes
			\item attributes associated with domains (variable types)
		\end{itemize}

	\subsection{Relational Schemata}

		Use of attributes creates relation schema such as:
		\\ MOVIE(title: $string$, production\_year: $number$)
		\\ \\
		\textbf{Relation Schema} provide abstract description of tuples in relation
		\\ \\
		\textbf{Database Schema} is set $S$ of relational schemata. Is basically the set of all tables and their attributes

	\subsection{Keys}
		Are used to uniquely identify tuples over all data in a given table.
		\\ They are used to restrict number of database instances, to something more realistic and identify objects efficiently
		\\ \\ 
		\textbf{Superkey} over relation schema is a subset of attributes that satisfies this uniqueness property
		\\ \textbf{Key} is a minimal superkey, is key if no other superkeys exist for R
		\\ \textbf{Foreign Key}: is a key used to index values from other values. Used to make reference between relational schemata. 
		\\ \tab - ensures referential integrity
		\\ \tab - no need to copy info from other tables
		\\ \tab - need to ensure that [x,y] $\subseteq$ [x,y] and not [y,x] (Order matters)
		\\ \\ \textbf{Example}
		\\ MOVIE(title: string, production\_year: number, director\_id: number) 
		\\ with key [title, production\_year]
		\\ \\ DIRECTOR(id: number, name: string)
		\\ with key [id]
		\\ \\ with foreign key: MOVIE[director\_id] $\subseteq$ DIRECTOR[id]

	\subsection{Integrity Constraints}
		\begin{itemize}
			\item Db schema should be meaningful, and satisfy all constraints
			\item should stay true to keys, and foreign keys
			\item constraints should interact with each other correctly
			\item should process queries and update efficiently
			\item should do this and make as few comprimises as possible
		\end{itemize}

\section{SQL as Data Definition and Manipulation Language}
	\subsection{General Information}
		\begin{itemize}
			\item \textbf{S}tructured \textbf{E}nglish \textbf{QUE}ry \textbf{L}anguage
			\item used as a standardised language in relational Db systems
			\item is query, data definition, and data management language
			\item names not case sensitive
		\end{itemize}

	\subsection{Keywords}
		\paragraph{CREATE} used to create tables. [CREATE TABLE $<tablename> <attribute specs>$]
		\paragraph{Attributes}: defined as [$<attribute\_name> <domain>$]
		\paragraph{NOT NULL}: used to ensure that attribute is not null.
		\paragraph{DROP TABLE}: removes relation schemata from Db system
		\paragraph{ALTER TABLE}: change existing relation schemata
		\\ \\ \textbf{CHARACTER} and \textbf{VARCHAR} are fixed or variable length strings. Variable length string are $VARCHAR(32)$
		\paragraph{NATIONAL CHARACTER}: string characters from other languages
		\\ \\ \textbf{INTEGER} and \textbf{SMALLINT} used for integers
		\\ \\ \textbf{NUMERIC} and \textbf{DECIMAL} used for fixed point numbers
		\\ \\ \textbf{FLOAT, REAL} and \textbf{DOUBLE PRECISION} used for floating point numbers
		\\ \\ \textbf{DATE} used for dates (year, month, day)
		\\ \\ \textbf{TIME} used for times (hour, minute, second)

	\subsection{Null and Duplicate Tuples}
		\paragraph{NULL} used to insert tuple where some of the information is not known. This is not good as it can create inconsistencies in information. SQL does this through use of a null marker

		\paragraph{Duplicate Tuples} are when another tuple exist with same values. NULL can make this confusing.
		\\ \tab - these duplicated really expensive to remove
		\\ \tab - relations do not contain these tuples

	\subsection{SQL Semantics}
		\paragraph{Table Schema} set of attributes
		\paragraph{Table over R} set of partial tuples
		\\ \\ These relations are idealised with no duplicates

	\subsection{Constraints}
		\begin{itemize}
			\item can add uniqueness parameters like primary keys on table
			\item NOT NULL makes sure that partial tuples don't exist
			\item \textbf{UNIQUE} $<attribute\_list>$ ensures that values for attributes exist only once
			\item \textbf{PRIMARY KEY} used on tables that are also UNIQUE, and ensures that NOT NULL applied to it too
			\item \mbox{$FOREIGN KEY <attributelist> REFERENCES <tablename> <attributelist>$}  used to create foreign key dependencies
			\item $CONSTRAINT <name> <constraint>$ \\used to give names to constraints
		\end{itemize}

	\subsection{Referential Actions}
		Are used to specify what happens if tuples updated or deleted
		\\ \\ SQL provies:
		\begin{itemize}
			\item \textbf{CASCADE} forces refercing tuples to be deleted
			\item \textbf{SET NULL} sets referencing tuples to NULL
			\item \textbf{SET DEFAULT} sets the tuple values to specified default
			\item \textbf{NO ACTION} does nothing
		\end{itemize}

		\paragraph{Example usage:}
		\mbox{FOREIGN KEY(title, year) REFERENCES MOVIE(title, year) ON DELETE \textbf{CASCADE}} \\ will delete director tuple if movie deleted
		\\\\ \mbox{FOREIGN KEY(title, year) REFERENCES MOVIE(title, year) ON DELETE \textbf{SET NULL}} \\ will keep tuple in director but won't know what he directed if movie deleted

	\subsection{CREATE statement}
		\begin{itemize}
			\item \textbf{CREATE ASSERTION} <name> CHECK defines integrity constraints
			\item \textbf{CREATE DOMAIN} define domains and check clause
			\item \textbf{CREATE VIEW $<name> \textbf{AS} <query>$} define views
			\item \textbf{CREATE GLOBAL TEMPORARY TABLE} makes a table for current SQL session
			\item \textbf{CREATE LOCAL TEMPORARY TABLE} makes table for current module or session
		\end{itemize}

	\subsection{Use as Data Manipulation Language}
		Allow insert, delete or update

		\begin{itemize}
			\item INSERT INTO $<tablename>$ VALUES $<tuple>$
			\item INSERT INTO $<tablename> <attributelist>$ VALUES $<tuples>$
			\item INSERT INTO $<tablename> <query>$
			\item DELETE FROM $<tablename>$ WHERE $<condition>$
			\item \mbox{UPDATE $<tablename>$ SET $<value-assignment-list>$ WHERE $<conditions>$}
		\end{itemize}

\section{Relational Query Languages: Algebra}
	\subsection{Query Languages}
		\begin{itemize}
			\item allow access and retrieval of data
			\item foundation based on logic, and allows for optimisation
			\item are not programming languages
			\item not intended for complex claculations
			\item easy efficient access to lots of data
		\end{itemize}

		\textbf{Query Language} must have
		\begin{itemize}
			\item formal language with
			\item input schema
			\item output schema
			\item query mapping
		\end{itemize}

		\textbf{Equivalent Queries}
		\begin{itemize}
			\item same input schema
			\item same output schema
			\item same query mapping
			\item if L $\sqsubseteq$ L' and L' $\sqsubseteq$ L
		\end{itemize}

		\textbf{Dominant}, if Q1 dominates Q2
		\begin{itemize}
			\item if L $\sqsubseteq$ L'
		\end{itemize}

	\subsection{Relational Algebra}
		\begin{itemize}
			\item useful for representing execution plan
			\item easily translated to SQL
			\item A is set of possible relations
			\item Set of Operations:
				\begin{itemize}
					\item Attribute Selection $\sigma$
					\\ \tab returns value where A in MOVIE is same as B in MOVIE 
					\\ \tab $[\sigma _{A=B}(MOVIE)]$

					\item Constant Selection $\sigma$
					\\ \tab returns vlaue where A is 3 in MOVIE 
					\\ \tab $[\sigma _{A=3}(MOVIE)]$

					\item Projection $\pi$
					\\ \tab returns title and date of MOVIE 
					\\ \tab $[\pi _{title, date}(MOVIE)]$

					\item Renaming $\delta \mapsto$
					\\ \tab renames attribute in MOVIE 
					\\ \tab $[\delta _{title \mapsto title'}(MOVIE)]$

					\item Union $\bigcup$
					\\ \tab takes 2 arguments and returns set of tuples contained in either result, removing duplicates
					\\ \tab $[\sigma _{A=2}(MOVIE) \bigcup \sigma_{A=3}(MOVIE)]$

					\item Difference $-$
					\\ \tab returns negation
					\\ \tab $[-\sigma _{A=3}(MOVIE)]$

					\item Natural Join $\bowtie$
					\\ \tab Joins two tables and return tuples that match in both tables
					\\ \tab $[ MOVIE \bowtie DIRECTOR ]$
				\end{itemize}
		\end{itemize}
		Redundant Operations
		\begin{itemize}
			\item Intersection $\bigcap$
			\\ \tab because can use DeMorgans to produce with $- and \bigcup$

			\item Cross Product $\times$
			\\ \tab because creates $N\times M$ tuples which is not space efficient

			\item Division $\div$
			\\ \tab because it uses cross product to generate all those that match
		\end{itemize}

		$\pi \sigma \bowtie \delta \bigcap \bigcup \bigvee \bigwedge \div \times - \in \mapsto$

	\subsection{Query Language $\mathcal{L} _{ALG}$ of Relational Algebra}
		wtf is this

	\subsection{Incremental Query Formulation}
		Can use multiple Queries to break up calculation to make it more readable. Ths will also reduce the amount of copying needed as you can use previous results in subsequent calculations.

\section{Relational Query Language: Calculus}
	\subsection{General}
		\begin{itemize}
			\item let users describe what they want
			\item simple to write query and translate to SQL
			\item sometimes relational algebra can get convulated
			\item based on first order logic
			\item if safe then can automatically be transformed to SQL style
			\item \textbf{Tuple Relational Calculus TRC} variables represent tuples
			\item \textbf{Domain Relational Calculus DRC} variables represent values in domain
		\end{itemize}

	\subsection{Object Properties}
		have variables or may set its value in query
		variable values must be in its domain
		can make complex properties by using negation, conjunction and exisential quantification
		remove brackets if doesn't increase ambiguity

	\subsection{Shortcuts}
		\begin{itemize}
			\item \textbf{Inequation} $A \neq B$ equivalent to $\neg A = B$
			\item \textbf{Disjunction} DeMorgan's: $A \vee B$ shortcuts $\neg(\neg A \wedge B)$
			\item \textbf{Universal Quantification} DeMorgan's: $\forall x (A)$ shortcuts $\neg \exists x (\neg A)$
			\item \textbf{Implication} $A \Rightarrow B $ shortcuts $ \neg A \vee B$
			\item \textbf{Equivalence} $A \Leftrightarrow B$ shortcuts $(A\Rightarrow B) \wedge (B \Rightarrow A)$ 
			\item \textbf{Successive Exestential Quantification} $\exists x_1 ( \exists x_2 (A))$ same as $\exists x_1,x_2(A)$
			\item \textbf{Successive Univeral Quantification} $\forall x_1 ( \forall x_2 (A))$ same as $\forall x_1,x_2(A)$
		\end{itemize}

	\subsection{Free variables}

	\subsection{Domain Relational Calculus}

	\subsection{Tuple Relational Calculus}

	\subsection{Safe Range Normal Form}






	$\neq \Leftrightarrow \Leftarrow \Rightarrow \forall \exists \neg x \vee \wedge$
		

\end{document}