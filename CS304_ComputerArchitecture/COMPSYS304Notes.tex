\documentclass{article} 
\newcommand\tab[1][0.5cm]{\hspace*{#1}}

\title{COMPSYS304 Notes 2017} 
\author{Theodore Oswandi} 

\usepackage[
	lmargin=2.5cm,
	rmargin=7cm,
	tmargin=1cm,
	bmargin=3cm,
	]{geometry}
\usepackage{enumitem}
\setlist{noitemsep}
\usepackage[none]{hyphenat}

\begin{document} \maketitle{} 

\section{Lecture 1}
	\subsection{Improvements}
		\begin{itemize}
			\item Semiconductor technology and computer architecture has increased lots in last 50 years
			\item Performance, which has also increased can be measured from standardised benchmarks
			\item Clock rate/frequency has also increased considerably in this time.
		\end{itemize}

		\paragraph{DBMS}(Database Management System): 
		\\ software package to store and manage databases

		\paragraph{Database System}: DBMS with database

		\paragraph{DBMS and uses}
		\begin{itemize}
			\item store large amounts of information
			\item code for queries
			\item protect from inconsistencies and crashes
			\item security
			\item concurrent access
		\end{itemize}


\end{document}

Lecture 1: 
	Semiconductor technology and computer architecture has increased lots in last 50 years
	Performance, which has also increased can be measured from standardised benchmarks
	Clock rate/frequency has also increased considerably in this time.

	Computer architecture
		ISA: Instruction set architecture. Boundary between hardware and software
		Organisation: high level computer design.
		Hardware: Logic, circuit-level design.

		Separate instruction set from implementation

	Memory Organisation
		See memory as single 1D array
		Address is index of this array, points to byte of memory.
		Memory Access Time: time to read data to/from memory

		Memory Speed != Processor speed.
		Fast memory is very expensive. Heirarchy used to maintain fluid functionality and keep things cheap.

		Register
			Smallest and fastest memory for CPU
			about 32-64 of them.
			Each are 32/64bits in size.
			Nanosecond access time

		Cache
			Slower than register
			8-256k
			Few nanoseconds access time
			Levels to this as well. L1, L2, L3 cache used in multiprocessor systems.

		Main Memory
			Slower than cache
			Megabytes to gigabytes of size.
			Tens of nanoseconds lookup time.

	ISA
		80x86
			old instruction set (1970s), but still used with extensions for things such as internet, tc...
			AMD/Intel both have same ISA but different implementation.

			Disadvantage
				power consumption is higher than things like iPad which use different ISA and consume a lot less power
				Also prevent some new innovation since it is so widely used in today's world.


		ISA Design
			something operands and opcode

			Stack Base Architecture
				Top of stack will contain result of operation.
				If receive ADD then processor knows next 2 inputs contain 2 numbers that need to be added.
				PUSH add something from stack.
				POP use value in stack.
				JVM designed to use Stack based architecture.

			Accumulator Based Archictecture.
				Using inputs from memory.
				Not used anymore today. Used in 1970s

			Register Memory Architecture
				Currently used today as x86
				Uses register for input as well as access values from memory.

			Register-Register Architecture
				Operands from register.
				LOAD and STORE from memory too.
				Need to specify destination register for output.

		A(1000) + B(2000) = C(3000)
			Stack Based Architecture
				PUSH 1000
				PUSH 2000
				ADD
				POP 3000

			Accumulator Based
				LOAD 1000
				ADD 2000
				STORE 3000

			Register Memory
				LOAD R2, 1000
				ADD R1, R2, 2000
				STORE R1, 3000

			Register Register
				LOAD R2, 1000
				LOAD R3, 2000
				ADD R1, R2, R3
				STORE R1, 3000


		ISA CLASSES
			RISC
				Reduced

			CISC
				Complex

			EPIC


		